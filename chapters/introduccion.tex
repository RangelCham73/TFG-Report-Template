\chapter{Introducción}
\label{ch:introduccion}

La introducción a este Proyecto de Fin de Grado (PFG) marca el comienzo de un viaje que ha sido concebido con el propósito de abordar una necesidad creciente en el ámbito del baloncesto: la mejora del rendimiento de los jugadores a través de la tecnología móvil. En un contexto donde el desarrollo atlético se fusiona cada vez más con la innovación tecnológica, surge la aplicación móvil \textit{Hoop Finder}, diseñada para proporcionar a los jugadores de baloncesto herramientas y recursos que optimicen su entrenamiento y potencien sus habilidades tanto individuales como colectivas.

La importancia de este proyecto radica en su capacidad para adaptarse a las demandas específicas de los usuarios, ofreciendo un enfoque interactivo y personalizado que se alinea con los avances en el análisis de datos deportivos. A través de la combinación de técnicas de análisis de datos y el aprovechamiento del GPS integrado en los dispositivos móviles, \textit{Hoop Finder} aspira a revolucionar la forma en que los jugadores monitorean y mejoran su desempeño en la cancha.

En este contexto, esta introducción busca generar expectativas al destacar la relevancia de la aplicación para el desarrollo del baloncesto, planteando interrogantes sobre cómo esta herramienta puede influir en la evolución del juego y en el crecimiento de los jugadores. Además, se establecerán los objetivos y alcances del proyecto, así como las suposiciones y limitaciones que guiarán el desarrollo y la evaluación de \textit{Hoop Finder}.

La finalidad principal de \textit{Hoop Finder} es proporcionar a los usuarios una plataforma que les permita alcanzar sus objetivos de mejora de manera efectiva y eficiente. Esto se logra mediante la implementación de funciones avanzadas, como el seguimiento mediante GPS, herramientas para el desarrollo del tiro y el entrenamiento en equipo. Al enfocarse en el diseño y desarrollo de una aplicación móvil que combine tecnología de vanguardia con las necesidades reales de los jugadores de baloncesto, se espera que \textit{Hoop Finder} contribuya significativamente al avance y la profesionalización de este deporte.

\section{App existentes}

Información recopilada de todas las apps existentes

\section{Conceptos básicos del baloncesto}

Para comprender adecuadamente el funcionamiento de la aplicación, es crucial tener claros los conceptos básicos de este deporte. El baloncesto es un juego de equipo que se practica en una cancha rectangular con dos aros en cada extremo. El objetivo principal es encestar la pelota en el aro contrario mientras se evita que el equipo rival haga lo mismo. Los jugadores pueden moverse driblando la pelota (botándola), pasarla entre sí y realizar tiros a la canasta. La defensa busca interceptar los pases, bloquear tiros y recuperar la posesión de la pelota. Es fundamental entender estos principios para aprovechar al máximo las funcionalidades y consejos que ofrece la aplicación de entrenamiento de baloncesto.

\section{Motivación}

Mi motivación para elegir este tema para mi PFG ha sido una combinación de intereses personales y profesionales. Desde hace muchos años, he sido un apasionado del baloncesto y he dedicado gran parte de mi tiempo libre al entrenamiento y participación en este deporte. Durante mi trayectoria como jugador, he sido testigo de las diversas áreas en las que la tecnología puede tener un impacto significativo en la mejora del rendimiento individual y colectivo en el baloncesto.

Además, como estudiante de ingeniería informática, siempre he estado fascinado por el potencial de las aplicaciones móviles para resolver problemas prácticos y mejorar la calidad de vida de las personas. La idea de desarrollar una aplicación móvil que combine mi pasión por el baloncesto con mi experiencia en tecnología móvil me pareció extremadamente emocionante y motivadora.

\section{Estructura de la memoria}

Cómo se organiza y estructura el proyecto en su totalidad. Esta sección presenta un resumen de los diferentes capítulos que conforman la memoria, así como una \textbf{muy breve} descripción de su contenido y propósito.
% hay que usar ref
Debéis utilizar los comandos \texttt{\textbackslash ref} o \texttt{\textbackslash cref} para referenciar a los capítulos, secciones, figuras, tablas, etc.
Por ejemplo, \enquote{En el capítulo \ref{ch:introduccion} se presenta la introducción del proyecto.} o \enquote{En la sección \ref{s:como-estructurar} se describen los objetivos del proyecto.}. Esto es importante para que el lector pueda navegar por la memoria de forma más sencilla.

Proporciona al lector una visión general de la estructura y el flujo del trabajo, permitiéndole comprender la secuencia lógica de cómo se desarrolla el trabajo o investigación.