\chapter{Objetivos}
\label{ch:Objetivos}

El TFG se centra en el desarrollo de una aplicación móvil de entrenamiento de baloncesto que utiliza la tecnología GPS para proporcionar una herramienta efectiva para mejorar las habilidades individuales y el rendimiento en equipo de los jugadores. Este proyecto busca integrar funciones de seguimiento de posición para ofrecer ejercicios personalizados en un campo de tiro, así como también facilitar la práctica de jugadas en equipo mediante la simulación de situaciones de juego reales. Los objetivos del TFG se dividen en generales y específicos, abarcando desde el diseño de la interfaz de usuario hasta la implementación de características que aseguren una experiencia de entrenamiento completa y efectiva

\section{General}

Los objetivos generales de este Trabajo de Fin de Grado se centran en el desarrollo de una aplicación móvil de entrenamiento de baloncesto que se destaque por su efectividad y accesibilidad para los usuarios. En primer lugar, se busca crear una aplicación que sea intuitiva y fácil de usar, con una interfaz de usuario amigable que permita a los jugadores acceder a las diferentes funcionalidades de entrenamiento de manera sencilla. Este aspecto es crucial para garantizar que la aplicación sea accesible para jugadores de todos los niveles de habilidad, desde principiantes hasta avanzados, y para que puedan aprovechar al máximo las herramientas de entrenamiento disponibles.

Además, se pretende integrar de manera efectiva la tecnología GPS en la aplicación, permitiendo el seguimiento preciso de la ubicación de los jugadores durante las sesiones de entrenamiento. Esto facilitará la personalización de los ejercicios según la posición del jugador en el campo, lo que garantizará un entrenamiento más específico y relevante para las necesidades individuales de cada usuario.

Por último, se aspira a fomentar el aprendizaje en equipo mediante la implementación de funciones que permitan practicar jugadas en conjunto, simulando situaciones de juego reales. Esto no solo mejorará la comprensión del juego por parte de los jugadores, sino que también promoverá la cooperación y la colaboración entre ellos, aspectos fundamentales en el desarrollo de habilidades de juego colectivo en el baloncesto.

En resumen, los objetivos generales de este proyecto se centran en crear una aplicación móvil de entrenamiento de baloncesto efectiva, accesible y que fomente el aprendizaje tanto a nivel individual como en equipo.

\section{Especifico}

Los objetivos específicos delineados para este proyecto de Trabajo de Fin de Grado se enfocan en aspectos concretos que contribuirán al logro de los objetivos generales establecidos. En primer lugar, se planea diseñar una interfaz de usuario que no solo sea intuitiva, sino que también ofrezca una experiencia fluida y atractiva para los usuarios. Esto implica no solo la disposición visual de los elementos en pantalla, sino también la implementación de controles y funciones que sean fáciles de entender y utilizar, independientemente del nivel de habilidad del usuario.

Además, se buscará integrar la tecnología GPS de manera efectiva en la aplicación, desarrollando algoritmos que permitan el seguimiento preciso de la ubicación de los jugadores en tiempo real. Aunque la aplicación no ajustará dinámicamente el entrenamiento, la tecnología GPS seguirá siendo fundamental para ubicar a los jugadores en un campo virtual y ofrecer ejercicios relevantes para diferentes posiciones en la cancha.

También se pretende desarrollar una amplia base de datos de ejercicios de entrenamiento, centrándose especialmente en el ataque, abarcando una variedad de habilidades como movimientos sin balón, selección de tiros y trabajo en equipo para generar oportunidades de anotación. Esta base de datos servirá como una fuente de recursos para los usuarios, permitiéndoles acceder a una amplia gama de opciones de entrenamiento y personalizar su experiencia según sus necesidades y objetivos específicos de ataque.

Finalmente, se explorará la implementación de herramientas de análisis de rendimiento que permitan a los usuarios evaluar su progreso y áreas de mejora en diferentes aspectos del juego ofensivo. Esto podría incluir métricas como la efectividad en la ejecución de jugadas, la precisión de los pases y la capacidad para crear oportunidades de anotación. En conjunto, estos objetivos específicos se diseñan para contribuir al desarrollo de una aplicación de entrenamiento de baloncesto integral y efectiva, que satisfaga las necesidades de los jugadores y promueva su mejora continua en el juego, especialmente en el aspecto ofensivo.