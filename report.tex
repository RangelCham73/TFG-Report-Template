\documentclass[%
    school=etsii,%
    type=pfg,%
    degree=GIC,%
    authorsex=m,%
    directorsex=m,%
]{urjc-report}

\addbibresource{references.bib}

\title{Aplicación movil de mejora de rendimiento de baloncesto}
\author{Diego Rangel Sanz}
\bibauthor{Proyecto, A.}
\director{Director Proyecto}
\bibdirector{Proyecto, D.}

\codirector[f]{Codirector Proyecto}{Proyecto, C.}

\abstract{spanish}{

    Este trabajo de fin de grado presenta el diseño y desarrollo de una aplicación móvil destinada a mejorar el rendimiento de los jugadores de baloncesto,
    denominada \textit{Hoop Finder}. La aplicación se enfoca en ofrecer herramientas y recursos para el entrenamiento físico, técnico y táctico,
    adaptados a las necesidades específicas de los usuarios. Se emplean técnicas de análisis de datos para proporcionar un enfoque interactivo y
    personalizado que motive a los jugadores a alcanzar sus objetivos de mejora.

    El desarrollo de la aplicación se basa en proporcionar herramientas para la mejora del tiro del jugador y entrenamiento en equipo. Además, mediante
    el uso del GPS, los jugadores podrán realizar un seguimiento de su entrenamiento. Se implementan características como un campo de tiro que rastrea
    la posición del jugador o jugadas de ataque en equipo. Además, se realiza un estudio estadístico para proporcionar al jugador toda la información
    necesaria para monitorizar sus entrenamientos.

    El resultado final es una aplicación móvil intuitiva y completa que ofrece a los jugadores de baloncesto una plataforma integral para mejorar su
    rendimiento individual y colectivo. Se espera que esta herramienta contribuya significativamente al desarrollo y perfeccionamiento de habilidades
    en jugadores de todos los niveles, desde amateurs hasta profesionales, y promueva una cultura de aprendizaje continuo en el ámbito del baloncesto.
}
\keywords{spanish}{Aplicación móvil para baloncesto; Mejora del tiro; Seguimiento mediante GPS; Rendimiento individual y colectivo}

\abstract{english}{
    This bachelor´s degree final project presents the design and development of a mobile application aimed at improving the performance of basketball players,
    named \textit{Hoop Finder}. The application focuses on providing tools and resources for physical, technical, and tactical training, tailored to the
    specific needs of users. Data analysis techniques are employed to provide an interactive and personalized approach that motivates players to
    achieve their improvement goals.

    The development of the application is based on providing tools for improving shooting skills and team training. Additionally, using GPS, players
    will be able to track their training. Features such as a shooting range that tracks the player's position or team attack plays
    are implemented. Furthermore, a statistical study is conducted to provide the player with all the necessary information to monitor their training.

    The result is an intuitive and comprehensive mobile application that offers basketball players a comprehensive platform to improve their
    individual and collective performance. It is expected that this tool will significantly contribute to the development and refinement of skills
    in players of all levels, from amateurs to professionals, and promote a culture of continuous learning in the field of basketball.
}
\keywords{english}{Basketball mobile application; Shooting improvement; GPS tracking; Individual and collective performance}

\acknowledgements{
    Aquí los agradecimientos que quieras dar. Y si no quieres, borras la entrada \texttt{\textbackslash acknowledgements} de \texttt{report.tex} y ya está.
}

\begin{document}

\include{frontmatter/glossary}

\chapter{Introducción}
\label{ch:introduccion}

La introducción a este Proyecto de Fin de Grado (PFG) marca el comienzo de un viaje que ha sido concebido con el propósito de abordar una necesidad creciente en el ámbito del baloncesto: la mejora del rendimiento de los jugadores a través de la tecnología móvil. En un contexto donde el desarrollo atlético se fusiona cada vez más con la innovación tecnológica, surge la aplicación móvil \textit{Hoop Finder}, diseñada para proporcionar a los jugadores de baloncesto herramientas y recursos que optimicen su entrenamiento y potencien sus habilidades tanto individuales como colectivas.

La importancia de este proyecto radica en su capacidad para adaptarse a las demandas específicas de los usuarios, ofreciendo un enfoque interactivo y personalizado que se alinea con los avances en el análisis de datos deportivos. A través de la combinación de técnicas de análisis de datos y el aprovechamiento del GPS integrado en los dispositivos móviles, \textit{Hoop Finder} aspira a revolucionar la forma en que los jugadores monitorean y mejoran su desempeño en la cancha.

En este contexto, esta introducción busca generar expectativas al destacar la relevancia de la aplicación para el desarrollo del baloncesto, planteando interrogantes sobre cómo esta herramienta puede influir en la evolución del juego y en el crecimiento de los jugadores. Además, se establecerán los objetivos y alcances del proyecto, así como las suposiciones y limitaciones que guiarán el desarrollo y la evaluación de \textit{Hoop Finder}.

La finalidad principal de \textit{Hoop Finder} es proporcionar a los usuarios una plataforma que les permita alcanzar sus objetivos de mejora de manera efectiva y eficiente. Esto se logra mediante la implementación de funciones avanzadas, como el seguimiento mediante GPS, herramientas para el desarrollo del tiro y el entrenamiento en equipo. Al enfocarse en el diseño y desarrollo de una aplicación móvil que combine tecnología de vanguardia con las necesidades reales de los jugadores de baloncesto, se espera que \textit{Hoop Finder} contribuya significativamente al avance y la profesionalización de este deporte.

\section{App existentes}

Información recopilada de todas las apps existentes

\section{Conceptos básicos del baloncesto}

Para comprender adecuadamente el funcionamiento de la aplicación, es crucial tener claros los conceptos básicos de este deporte. El baloncesto es un juego de equipo que se practica en una cancha rectangular con dos aros en cada extremo. El objetivo principal es encestar la pelota en el aro contrario mientras se evita que el equipo rival haga lo mismo. Los jugadores pueden moverse driblando la pelota (botándola), pasarla entre sí y realizar tiros a la canasta. La defensa busca interceptar los pases, bloquear tiros y recuperar la posesión de la pelota. Es fundamental entender estos principios para aprovechar al máximo las funcionalidades y consejos que ofrece la aplicación de entrenamiento de baloncesto.

\section{Motivación}

Mi motivación para elegir este tema para mi PFG ha sido una combinación de intereses personales y profesionales. Desde hace muchos años, he sido un apasionado del baloncesto y he dedicado gran parte de mi tiempo libre al entrenamiento y participación en este deporte. Durante mi trayectoria como jugador, he sido testigo de las diversas áreas en las que la tecnología puede tener un impacto significativo en la mejora del rendimiento individual y colectivo en el baloncesto.

Además, como estudiante de ingeniería informática, siempre he estado fascinado por el potencial de las aplicaciones móviles para resolver problemas prácticos y mejorar la calidad de vida de las personas. La idea de desarrollar una aplicación móvil que combine mi pasión por el baloncesto con mi experiencia en tecnología móvil me pareció extremadamente emocionante y motivadora.

\section{Estructura de la memoria}

Cómo se organiza y estructura el proyecto en su totalidad. Esta sección presenta un resumen de los diferentes capítulos que conforman la memoria, así como una \textbf{muy breve} descripción de su contenido y propósito.
% hay que usar ref
Debéis utilizar los comandos \texttt{\textbackslash ref} o \texttt{\textbackslash cref} para referenciar a los capítulos, secciones, figuras, tablas, etc.
Por ejemplo, \enquote{En el capítulo \ref{ch:introduccion} se presenta la introducción del proyecto.} o \enquote{En la sección \ref{s:como-estructurar} se describen los objetivos del proyecto.}. Esto es importante para que el lector pueda navegar por la memoria de forma más sencilla.

Proporciona al lector una visión general de la estructura y el flujo del trabajo, permitiéndole comprender la secuencia lógica de cómo se desarrolla el trabajo o investigación.
\chapter{Objetivos}
\label{ch:Objetivos}

El TFG se centra en el desarrollo de una aplicación móvil de entrenamiento de baloncesto que utiliza la tecnología GPS para proporcionar una herramienta efectiva para mejorar las habilidades individuales y el rendimiento en equipo de los jugadores. Este proyecto busca integrar funciones de seguimiento de posición para ofrecer ejercicios personalizados en un campo de tiro, así como también facilitar la práctica de jugadas en equipo mediante la simulación de situaciones de juego reales. Los objetivos del TFG se dividen en generales y específicos, abarcando desde el diseño de la interfaz de usuario hasta la implementación de características que aseguren una experiencia de entrenamiento completa y efectiva

\section{General}

Los objetivos generales de este Trabajo de Fin de Grado se centran en el desarrollo de una aplicación móvil de entrenamiento de baloncesto que se destaque por su efectividad y accesibilidad para los usuarios. En primer lugar, se busca crear una aplicación que sea intuitiva y fácil de usar, con una interfaz de usuario amigable que permita a los jugadores acceder a las diferentes funcionalidades de entrenamiento de manera sencilla. Este aspecto es crucial para garantizar que la aplicación sea accesible para jugadores de todos los niveles de habilidad, desde principiantes hasta avanzados, y para que puedan aprovechar al máximo las herramientas de entrenamiento disponibles.

Además, se pretende integrar de manera efectiva la tecnología GPS en la aplicación, permitiendo el seguimiento preciso de la ubicación de los jugadores durante las sesiones de entrenamiento. Esto facilitará la personalización de los ejercicios según la posición del jugador en el campo, lo que garantizará un entrenamiento más específico y relevante para las necesidades individuales de cada usuario.

Por último, se aspira a fomentar el aprendizaje en equipo mediante la implementación de funciones que permitan practicar jugadas en conjunto, simulando situaciones de juego reales. Esto no solo mejorará la comprensión del juego por parte de los jugadores, sino que también promoverá la cooperación y la colaboración entre ellos, aspectos fundamentales en el desarrollo de habilidades de juego colectivo en el baloncesto.

En resumen, los objetivos generales de este proyecto se centran en crear una aplicación móvil de entrenamiento de baloncesto efectiva, accesible y que fomente el aprendizaje tanto a nivel individual como en equipo.

\section{Especifico}

Los objetivos específicos delineados para este proyecto de Trabajo de Fin de Grado se enfocan en aspectos concretos que contribuirán al logro de los objetivos generales establecidos. En primer lugar, se planea diseñar una interfaz de usuario que no solo sea intuitiva, sino que también ofrezca una experiencia fluida y atractiva para los usuarios. Esto implica no solo la disposición visual de los elementos en pantalla, sino también la implementación de controles y funciones que sean fáciles de entender y utilizar, independientemente del nivel de habilidad del usuario.

Además, se buscará integrar la tecnología GPS de manera efectiva en la aplicación, desarrollando algoritmos que permitan el seguimiento preciso de la ubicación de los jugadores en tiempo real. Aunque la aplicación no ajustará dinámicamente el entrenamiento, la tecnología GPS seguirá siendo fundamental para ubicar a los jugadores en un campo virtual y ofrecer ejercicios relevantes para diferentes posiciones en la cancha.

También se pretende desarrollar una amplia base de datos de ejercicios de entrenamiento, centrándose especialmente en el ataque, abarcando una variedad de habilidades como movimientos sin balón, selección de tiros y trabajo en equipo para generar oportunidades de anotación. Esta base de datos servirá como una fuente de recursos para los usuarios, permitiéndoles acceder a una amplia gama de opciones de entrenamiento y personalizar su experiencia según sus necesidades y objetivos específicos de ataque.

Finalmente, se explorará la implementación de herramientas de análisis de rendimiento que permitan a los usuarios evaluar su progreso y áreas de mejora en diferentes aspectos del juego ofensivo. Esto podría incluir métricas como la efectividad en la ejecución de jugadas, la precisión de los pases y la capacidad para crear oportunidades de anotación. En conjunto, estos objetivos específicos se diseñan para contribuir al desarrollo de una aplicación de entrenamiento de baloncesto integral y efectiva, que satisfaga las necesidades de los jugadores y promueva su mejora continua en el juego, especialmente en el aspecto ofensivo.
\chapter{Desarrollo del Software}
\label{ch:Desarrollo del Software}

Escribir todo el proceso de desarrollo realizado

\section{Metodologia}

Metodologia seguida en el desarrollo en este caso la metodologia agil scrum

\section{Prototipos}

Explicar toda la fase del desarrollo del diseño y prototipos de la aplicación

\section{Herramientas utilizadas}

GitHub, Figma, IntelliJ...
\chapter{Resultados}
\label{ch:Resultados}

Escribir los resultados conseguidos en el desarrollo y explicar el funcionamiento final de la aplicación
\chapter{Conclusiones y trabajos futuros}
\label{ch:Conclusiones y trabajos futuros}

Conclusiones finales del PFG
\chapter{Anexo}
\label{ch:Anexo}
\input{chapters/configuracion}
\input{chapters/componentes}
\input{chapters/licencia}

\appendix

\input{appendices/escuelas-y-titulos}
\input{appendices/ampliar}
\input{appendices/paquetes}

\end{document}
